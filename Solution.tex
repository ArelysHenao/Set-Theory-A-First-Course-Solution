\documentclass[12pt]{article}
\usepackage{amsmath, amssymb}
\usepackage{geometry}
\usepackage{fancyhdr}
\usepackage{parskip}

\geometry{a4paper, margin=1in}
\pagestyle{fancy}
\fancyhf{}
\rhead{Exercise 1.1}
\lhead{Set Theory}
\rfoot{\thepage}

\title{\textbf{Set Theory – Exercise 1.1}}
\author{}
\date{}

\begin{document}

\maketitle

\section*{Definition}

Let \( A \) and \( B \) be sets:

\begin{enumerate}
    \item \( \neg(a \in A) \equiv a \notin A \)
    \item \( a \in A \setminus B \equiv a \in A \land a \notin B \)
    \item \( a \in A \cup B \equiv a \in A \lor a \in B \)
    \item \( a \in A \cap B \equiv a \in A \land a \in B \)
\end{enumerate}

\section*{Exercise 1.1}

Let \( A, B, C \) be sets.

\subsection*{Problem}

If \( a \notin A \setminus B \) and \( a \in A \), show that \( a \in B \).

\subsection*{Proof}

We proceed step by step, transforming the left-hand side:

\begin{align*}
    & a \notin A \setminus B \land a \in A \\
    \equiv{} & \neg(a \in A \setminus B) \land a \in A \quad \text{(Definition of } \notin) \\
    \equiv{} & \neg(a \in A \land a \notin B) \land a \in A \quad \text{(Definition of } \setminus) \\
    \equiv{} & (\neg(a \in A) \lor \neg(a \notin B)) \land a \in A \quad \text{(De Morgan's Law)} \\
    \equiv{} & (\neg(a \in A) \lor \neg(\neg(a \in B))) \land a \in A \quad \text{(Definition of } \notin) \\
    \equiv{} & (\neg(a \in A) \lor a \in B) \land a \in A \quad \text{(Double Negation)} \\
    \equiv{} & (\neg(a \in A) \land a \in A) \lor (a \in B \land a \in A) \quad \text{(Distributive Law)} \\
    \equiv{} & \text{false} \lor (a \in B \land a \in A) \quad \text{(Contradiction: } A \land \neg A) \\
    \equiv{} & a \in B \land a \in A \quad \text{(Identity of } \lor) \\
    \Rightarrow{} & a \in B \quad \text{(Weakening)}
\end{align*}

\hfill\ensuremath{\blacksquare}

\end{document}
